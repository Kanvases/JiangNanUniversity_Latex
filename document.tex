\documentclass[12pt]{article}
\usepackage{xeCJK}
\usepackage{graphicx} 
\usepackage{geometry}
\geometry{a4paper,left=2cm,right=2cm,top=2cm,bottom=2cm}
\pagestyle{plain}
\usepackage{array} 
\usepackage{type1cm}
%这里可能要改!取决于宋体在你系统上的名字。以下是MacOS
\setCJKmainfont{Songti SC}

% 题目后的下划线,自定义粗细。使用方法:\hlinew{xxpt}
\makeatletter
\def\hlinew#1{
	\noalign{\ifnum0=`}\fi\hrule \@height #1 \futurelet
	\reserved@a\@xhline}
\makeatother


\begin{document}
	%---------------------封面---------------------------%
	% 江南大学 图片
	\begin{figure}
		\centering	
		\includegraphics[width=0.4\textwidth]{JiangNanName}
	\end{figure}
	
	%大标题
	\begin{center}
		% 32pt 字号 48pt 1.5倍行距(32x1.5=48)
		\fontsize{32pt}{48pt} \selectfont \textbf{\\《什么什么课》\\课程设计\\}
	\end{center}
	\vspace{1cm}
	% 二标题  请不要在下行后加回车。
	\mbox{\fontsize{22pt}{22pt} \selectfont\textbf{\\题目:}}
	% 这里写自己的标题。 [t] 保证下划线与“题目:”对齐 	不要在上一行加回车
	%二标题向左对齐
	%\begin{tabular}[t]{p{0.75\textwidth}}
	%二标题居中
	\begin{tabular}[t]{p{0.75\textwidth}<{\centering}}
		\fontsize{22pt}{22pt} \selectfont \textbf{这里可以写下一行的标题}\\
		\hlinew{1pt}
		\\
		\fontsize{22pt}{22pt} \selectfont \textbf{需要手动换下一行}\\
		\hlinew{1pt}
	\end{tabular}

	\vspace{2cm}
	
	%院系
	\begin{center}
	 	\fontsize{18pt}{18pt} \selectfont\underline{ 数字媒体 }学院 \underline{ 数字媒体技术 } 专业
	\end{center}

	\vspace{2cm}
	
	%学生信息
	\begin{center}
	\fontsize{18pt}{18pt} \selectfont
	\begin{tabular}[t]{cp{0.4\textwidth}<{\centering}}
		
		学\hphantom{文本}号&1030516XXX\\
		\cline{2-2}
		\\
		班\hphantom{文本}级&1602\\
		\cline{2-2}
		\\
		学生姓名&姓名\\
		\cline{2-2}
		\\
		导师姓名&姓名\\
		\cline{2-2}
	\end{tabular}
	\end{center}

	\vspace{2cm}
	
	%日期
	\begin{center}
		\fontsize{18pt}{18pt} \selectfont 二〇一八年十二月
	\end{center}
%---------------------封面完了---------------------------%


\newpage
\section {系统创新点}
\section {设计的目的与要求}
\section {总体方案设计}
\section {各个功能模块的主要实现程序}
\section {课程设计总结与体会}
\section {参考文献}
\end{document}
